\section{Appendix}\label{sec:appendix}
\framecard{\insertsection}
\subsection{Appendix A - Matrix Calculus}

\begin{frame}{\insertsubsection}

\begin{definition}[Matrix Multiplication]
Given $\mathbf{A}$ being $m \times n$ and $\mathbf{B}$ being $p \times q$
\begin{equation*}
\mathbf{A}\mathbf{B} = \left[ \sum^n_{s=1} a_{is}b_{sj} \right] \text{, with } n = p 
\end{equation*}
\begin{equation*}
\mathbf{B}\mathbf{A} = \left[ \sum^r_{k=1} b_{ik}a_{kj} \right] \text{, with } m = q
\end{equation*}
\end{definition}

\end{frame}

\begin{frame}{\insertsubsection}

\begin{definition}[Matrix Multiplication]
Given $\mathbf{A}$ being $m \times n$ and $\mathbf{B}$ being $p \times q$
\begin{equation*}
\left[ \mathbf{A}\mathbf{B} \right]^T = \left[ \sum^m_{s=1} a_{is}b_{sj} \right]^T  = \left[ \sum^n_{s=1} b_{is}a_{sj} \right] = \mathbf{B}^T \mathbf{A}^T \text{, with } n = p
\end{equation*}
\end{definition}

\end{frame}


\begin{frame}{\insertsubsection}

\begin{block}{Proposition}
Given $\mathbf{y}$ being $m \times 1$, $\mathbf{x}$ being $n \times 1$, $\mathbf{A}$ being $m \times n$ independent of $\mathbf{x}$ and

\begin{equation*}
\mathbf{y} = \mathbf{A} \mathbf{x}
\end{equation*}

Then

\begin{equation*}
\frac{\partial \mathbf{y}}{\partial \mathbf{x}} = \mathbf{A}
\end{equation*}

\end{block}

\end{frame}

\begin{frame}{\insertsubsection}

\begin{definition}[Matrix Derivative]
Given $\mathbf{A}$ being $m \times n$ and $\mathbf{B}$ being $p \times q$
\begin{equation*}
\mathbf{A}\mathbf{B} = \left[ \sum^n_{s=1} a_{is}b_{sj} \right]
\end{equation*}
\begin{equation*}
\mathbf{B}\mathbf{A} = \left[ \sum^r_{k=1} b_{ik}a_{kj} \right]
\end{equation*}
\end{definition}

\end{frame}

\subsection{Appendix B - Bayes' Rule}

\begin{frame}{\insertsubsection}

\begin{definition}[Bayes' Rule]
Given a dataset $\mathcal{D} := \{ \mathbf{x},\mathbf{t} \}$ and $\mathbf{w}$ being $p \times q$
\begin{equation*}
\mathbf{A}\mathbf{B} = \left[ \sum^n_{s=1} a_{is}b_{sj} \right] \text{, with } n = p 
\end{equation*}
\begin{equation*}
\mathbf{B}\mathbf{A} = \left[ \sum^r_{k=1} b_{ik}a_{kj} \right] \text{, with } m = q
\end{equation*}
\end{definition}

\end{frame}