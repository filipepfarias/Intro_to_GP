\section{Probability and Random Process Theory Review}
\framecard{\insertsection}
\subsection{Basic Concepts of Probability Theory}


\begin{frame}{Specifying Random Experiments}

Let's start with the sentence

\begin{block}{Sentence}
\textit{A \textbf{random experiment} is specified by stating an \textbf{\textcolor{red}{experimental procedure}} and a \textbf{\textcolor{red}{set}} of one or more measurements or observations.}
\end{block}

\end{frame}

\subsection{The Gaussian distribution}
\begin{frame}{\insertsubsection}
The \textbf{Gaussian distribution} is defined as
	\begin{equation}\label{eq:gaussian-distribution}
	\mathcal{N}(x|\mu,\sigma^2) = \frac{1}{(2\pi\sigma^2)^{1/2}}\exp\left\{-\frac{1}{2\sigma^2}(x-\mu)^2\right\}
	\end{equation}
\end{frame}


\subsection{Independency of two random variables}
\begin{frame}{\insertsubsection}
\visible<2->{
\begin{block}{Sentence}
\textbf{X and Y are independent random variables} if \textit{any} event $A_1$ defined in terms of $X$ is independent of \textit{any} event $A_2$ defined in terms of Y
\end{block}
}
\visible<3->{
The sentence above is equivalent to say mathematically that
}
\visible<4->{
	\begin{equation}
	P[X \text{ in } A_1, Y \text{ in } A_2]=P[X \text{ in } A_1]P[ Y \text{ in } A_2]
	\end{equation}
}
\visible<5->{
that means in other words that \textit{if $X$ and $Y$ are independent discrete random variables, then the \textbf{joint probability mass function (pmf)} is equal to the product of the marginal pmf's.}
}
\end{frame}

%\begin{frame}{Overleaf users}
%
%\begin{alertblock}{Warning}
%You can ignore this slide if you're \textbf{not} working with Overleaf.
%\end{alertblock}
%
%\vskip 0.5cm
%
%Overleaf, Beamer and Biber do not always get along well together. For this reason, if you make a mistake while writing this presentation, in the drop-down error message you'll \textbf{always} get Biber-related error messages.
%
%\vskip 0.5cm
%
%Luckily, you just have to click on ``\texttt{go to first error/warning}'' and the UI will scroll to the line containing your mistake.
%
%\end{frame}
%
%\begin{frame}[fragile]
%\frametitle{Compiling}
%
%\begin{alertblock}{Warning}
%You can ignore this slide if you're working with Overleaf.
%\end{alertblock}
%
%To compile this deck you'll need the \texttt{biber} package. Probably your \TeX editor already supports it; if not, you will easily find online the instructions to install it.
%
%\vskip 0.5cm
%
%If you're not using an editor, you can compile this presentation using the command line by running:
%
%\begin{verbatim}
%$ pdflatex main.tex
%$ biber main.bcf
%$ pdflatex main.tex
%$ pdflatex main.tex
%\end{verbatim}
%
%
%\end{frame}
