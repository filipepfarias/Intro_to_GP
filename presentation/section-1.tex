\section{Probability and Random Process Theory Review}
\subsection{Basic Concepts of Probability Theory}
\framecard{\insertsection}


\begin{frame}{Specifying Random Experiments}

Let's start with the sentence

\begin{block}{Sentence}
\textit{A \textbf{random experiment} is specified by stating an \textbf{\textcolor{red}{experimental procedure}} and a \textbf{\textcolor{red}{set}} of one or more measurements or observations.}
\end{block}

\end{frame}



\begin{frame}{Overleaf users}

\begin{alertblock}{Warning}
You can ignore this slide if you're \textbf{not} working with Overleaf.
\end{alertblock}

\vskip 0.5cm

Overleaf, Beamer and Biber do not always get along well together. For this reason, if you make a mistake while writing this presentation, in the drop-down error message you'll \textbf{always} get Biber-related error messages.

\vskip 0.5cm

Luckily, you just have to click on ``\texttt{go to first error/warning}'' and the UI will scroll to the line containing your mistake.

\end{frame}

\begin{frame}[fragile]
\frametitle{Compiling}

\begin{alertblock}{Warning}
You can ignore this slide if you're working with Overleaf.
\end{alertblock}

To compile this deck you'll need the \texttt{biber} package. Probably your \TeX editor already supports it; if not, you will easily find online the instructions to install it.

\vskip 0.5cm

If you're not using an editor, you can compile this presentation using the command line by running:

\begin{verbatim}
$ pdflatex main.tex
$ biber main.bcf
$ pdflatex main.tex
$ pdflatex main.tex
\end{verbatim}


\end{frame}
