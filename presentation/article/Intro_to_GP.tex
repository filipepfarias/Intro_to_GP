% !TeX root = ./Intro_to_GP.tex
\documentclass[11pt]{article} % For LaTeX2e
\usepackage{Intro_to_GP}

\title{Introduction to Gaussian Processes}

\IntroGPfinaltrue

\author{
Filipe P.~Farias \\
Teleinformatics Engineering Department\\
Federal University of Ceará\\
\texttt{filipepfarias@fisica.ufc.br} \\
}

% The \author macro works with any number of authors. There are two commands
% used to separate the names and addresses of multiple authors: \And and \AND.
%
% Using \And between authors leaves it to \LaTeX{} to determine where to break
% the lines. Using \AND forces a linebreak at that point. So, if \LaTeX{}
% puts 3 of 4 authors names on the first line, and the last on the second
% line, try using \AND instead of \And before the third author name.

\newcommand{\fix}{\marginpar{FIX}}
\newcommand{\new}{\marginpar{NEW}}

\begin{document}

\maketitle

\begin{abstract}
   A wide variety of methods exists to deal with supervised learning, as restrict a class of linear functions of the inputs, as linear regression, or give a prior probability to every possible function, giving high probability to the functions we consider more likely. The second approach is a way to Gaussian process itself. We will make the pathway through a intuitive construction of this framework.
\end{abstract}

\section{Introduction}

\lipsum[1]%

\section{Linear Regression}

Starting with a simple regression problem. Be the dataset $\mathrm{D}=\left\{ x_i,y_i|i=1,\dots,N \right\}$, where we observe a real-valued input variable $x$ and a measured real-valued variable $y$. Then, we'll use synthetically generated data for comparison against any learned \textit{model}. And $N$ will be the number od observations of the value $y$. Our objetive is make predictions of the new value $\hat{y}$ for some new input $\hat{x}$.

For this example, we'll use a simple approach based on curve fitting by the polynomial model, i.e, being the function

\begin{equation}
   f(x,\mathbf{w}) = \sum_{j=0}^M w_j x^j
\end{equation}

where $M$ is the order of the polynomial and $\mathbf{w}=\left[ w_0,\dots,w_M \right]$ its coefficients. It's important to note that the $f$ isn't linear in $x$ but in $\mathbf{w}$. These functions which are linear on the unknown
% \begin{figure}[H]
%    \centering
%    \includegraphics[width=.25\textwidth]{Figures/f1.png}
%    \caption{Símbolos do diagrama de blocos. (a) Adição de duas seqüências. (b) Multiplicação de uma sequência por um  constante. c) atraso da unidade.}
%    \label{graph:A-and-B-systems}
% \end{figure}

\lipsum[1]

\end{document}
