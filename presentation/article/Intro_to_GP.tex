% !TeX root = ./Intro_to_GP.tex
\documentclass[11pt]{article} % For LaTeX2e
\usepackage{Intro_to_GP}

\title{Introduction to Gaussian Processes}

\IntroGPfinaltrue

\author{
Filipe P.~Farias \\
Teleinformatics Engineering Department\\
Federal University of Ceará\\
\texttt{filipepfarias@fisica.ufc.br} \\
}

% The \author macro works with any number of authors. There are two commands
% used to separate the names and addresses of multiple authors: \And and \AND.
%
% Using \And between authors leaves it to \LaTeX{} to determine where to break
% the lines. Using \AND forces a linebreak at that point. So, if \LaTeX{}
% puts 3 of 4 authors names on the first line, and the last on the second
% line, try using \AND instead of \And before the third author name.

\newcommand{\fix}{\marginpar{FIX}}
\newcommand{\new}{\marginpar{NEW}}

\begin{document}

\maketitle

% \begin{abstract}
%    A stochastic process is a generalization of a probability distribution (which describes a finite-dimensional random variable) to \textit{functions}. By focussing on processes which are \textit{Gaussian}, it turns out that the computations required for inference and learning become relatively easy. Thus, the supervised learning problems in machine learning which can be thought of as learning a function from examples can be cast directly into the Gaussian process framework \cite{Rasmussen:2005:GPM:1162254}.
% \end{abstract}

\section{Introduction}
\nocite{hennig2013gaussian}

The Gaussian processes has been shown as a powerfull framework for robust estimation, a flexible non-linear \textit{regression} \cite{Rasmussen_2004}, case which it will be the main object of this work. First, we`ll see some initial concept that will set the background for the GP.

Let`s suppose a data set $\mathcal{D} = \left\{ \mathbf{x}_i,\mathbf{t}_i \right\}_1^N$ which we denote the input as $\mathbf{x}$, and the output (or target) as $t$ and $N$ as the number of observations. The first step in our workflow is define a \textit{training} set, i.e. some data that is given to make our first assumptions of the \textit{model}. The model is defined as a guess of a law that rules the phenomenon of which our data was observed. This law can be, by example a senoid function as represented in \textcolor{red}{Figure 1}. Given this training data we wish to make \textit{predictions} for new inputs $\mathbf{x_*}$ that we have not seen in the training set.

The model may or not contain some \textit{parameters} that will be adjusted during the \textit{training phase}, where the model itself or its parameters are modified aiming to reduce the mismatch with the train set. In general we define a \textit{loss function} which increases with the mismatch, or \textit{error} of the model. Then our work will be reduce this error. The model with the least error will be the one that defines when our model has learned.

Unfortunately, in the trying of obtain the best model, we may lose the capability of generalize our model, then we can increase the flexibility of it by accepting some \textit{uncertainty} above it. More, sometimes a good first assumption can make the difference to the estimation, and one may want to put its beliefs in the model, before observe some data, i.e. make some \textit{prior} assumptions. These both strategies are well defined by the Bayesian statistics and could help if we assume a probabilistic model.

% Thus we need to move from the finite training data $\mathcal{D}$ to a function $f$ that makes predictions for all possible input values. To do this we must make assumptions about the characteristics of the underlying function, as otherwise any function which is consistent with the training data would be equally valid.




The approach is to give a \textit{prior} probability to every possible function, where higher probabilities are given to functions that we consider to be \textit{more likely}, for example because they are smoother than other functions. \cite{Rasmussen:2005:GPM:1162254}. This appears to have a serious problem, in that surely there are an infinite set of possible functions to compute in a finite time. This is where the Gaussian \textit{process} comes as a possible approach. 
A Gaussian process is a generalization of the Gaussian probability \textit{distribution}. Whereas a probability distribution describes random variables which are scalars or vectors (for multivariate distributions), a random process governs the properties of functions \cite{getoor2009}. 

We can think of a function as a very long vector, each entry in the vector specifying the function value $f(x)$ at a particular input $x$. Indeed, the question of how we deal computationally with these infinite dimensional objects has the resolution: if you ask only for the properties of the function at a finite number of points, then \textit{infer} the properties for the predictions in the Gaussian process will give you the same answer if you ignore the infinitely many other points, as if you would have taken them all into account. One of the main attractions of the Gaussian process framework is precisely that it unites a sophisticated and consistent view with computational tractability.

\section{Linear Regression}

To understand the concepts the are covered by the Gaussian processes, we start with a simple regression problem. Be the data set $\mathcal{D}=\left\{ x_i,t_i|i=0,\dots,N-1 \right\}$, where we observe a real-valued input variable $x$ and a measured real-valued variable $t$. Then, we'll use synthetically generated data for comparison against any learned \textit{model}. And $N$ will be the number of observations of the value $t$. Our objective is make predictions of the new value $\hat{t}$ for some new input $\hat{x}$.

\begin{figure}[htbp]
   \centering
   \includegraphics{graphics/prml/{Figure1.2}.pdf}
   \caption{Training data set with $n=10$ points in blue. The green curve shows the function $sin(2\pi x)$ used to generate the data. Our goal is to predict the value of t for some new value of x, without knowledge of the green curve \cite{Bishop:2006:PRM:1162264}.}
   % \label{<label>}
\end{figure}

For this example, we'll use a simple approach based on curve fitting by the polynomial model, i.e., being the function

\begin{equation}
   \label{eq:lin-reg-model-1}
   y(x,\mathbf{w}) = \sum_{j=0}^{M-1} w_j x^j
\end{equation}

where $M$ is the order of the polynomial and $\mathbf{w}=\left[ w_0,\dots,w_M \right]$ its coefficients. It's important to note that the $y$ isn't linear in $x$ but in $\mathbf{w}$. These functions which are linear on the unknown parameters are called \textit{linear models}.

We can extend the class of models considering linear combinations of nonlinear functions of the input variables, i.e.

\begin{equation}
   y(x,\mathbf{w}) = \sum_{j=0}^{M-1} \phi_j (x) w_j 
   \label{eq:lin-reg-model-phi}
\end{equation}

where $\phi_j (x)$ are known as \textit{basis functions}, and then the total number of parameters for this model will be $M$. In the example of the curve fitting, the polynomial regression implies that $\phi_j(x)=x^j$. We can evaluate the same operation of (\ref{eq:lin-reg-model-1}) in the matrix form by

\begin{equation}
   y(x,\mathbf{w}) = \boldsymbol{\phi}(x)^\top  \mathbf{w}
   \label{eq:lin-reg-model-1-matrix-form}
\end{equation}

where $\boldsymbol{\phi}(x) = \left[ \phi_0(x), \dots, \phi_{M-1}(x) \right]^\top$. 

\begin{figure}[htbp]
   \centering
   \includegraphics{graphics/prml/{Figure1.3}.pdf}
   \caption{The error function (\ref{eq:lin-reg-error-function-1}) corresponds to (one half of) the sum of the squares of the displacements (shown by the vertical green bars) of each data point from the function $y(x, \mathbf{w})$ \cite{Bishop:2006:PRM:1162264}.}
   % \label{<label>}
\end{figure}

The values of $\mathbf{w}$ are obtained by minimizing the \textit{error function}, a measure of the distance between the training data set and $y$, given values of $\mathbf{w}$. By the way, the chosen error function will be 

\begin{equation}
   \label{eq:lin-reg-error-function-1}
   E(\mathbf{w}) = \frac{1}{2} \sum_{n=1}^N \left\{ y(x_i,\mathbf{w})-y_i \right\}^2
\end{equation}

This indicate that if $E$ is zero, $y$ passes exactly through each training data point. Observe that $E$ do not assume negative values because of its quadratic form, then we can find $\mathbf{w}$ by finding the minimum value of $E$, denoted $\mathbf{w^*}$, by

\begin{equation}
   \frac{\partial E}{\partial \mathbf{w}} = 0
\end{equation}

We can rewrite (\ref{eq:lin-reg-error-function-1}) in the matrix form, as $\mathbf{y} = y(\mathbf{x},\mathbf{w})$, where $\mathbf{x}=\left[ x_0,\dots, x_{N-1} \right]^\top$, i.e. $y$ evaluated for all input variables (See Appendix \ref{subsec:app-matrix-form}). Then we have

\begin{equation}
   \label{eq:lin-reg-error-function-matrix-form}
   \mathbf{y} = \Phi \mathbf{w}
\end{equation}

where $\Phi$ is the \textit{design matrix} such that $\boldsymbol{\phi}(x)$ is evaluated for all $\mathbf{x}$. Proceeding with the minimization, we obtain the optimal $\mathbf{w}$, or $\mathbf{w}^*$ by

\begin{equation}
   \label{eq:lin-reg-opt-param}
   \mathbf{w}^{*} = (\Phi^\top \Phi)^{-1}\Phi^\top \mathbf{t}
\end{equation}

which are the parameters that best fit the model to the data.

\begin{figure}
   \centering
   \begin{subfigure}[htpb]{0.45\linewidth}
       \centering
       \includegraphics[width=1\linewidth]{graphics/prml/{Figure1.4a}.pdf}
   \end{subfigure}
   \hfill
   \begin{subfigure}[htpb]{0.45\linewidth}
       \centering
       \includegraphics[width=1\linewidth]{graphics/prml/{Figure1.4b}.pdf}       
   \end{subfigure}
   \hfill \\
   \begin{subfigure}[htpb]{0.45\linewidth}
       \centering
       \includegraphics[width=1\linewidth]{graphics/prml/{Figure1.4c}.pdf}
   \end{subfigure}
   \hfill 
   \begin{subfigure}[htpb]{0.45\linewidth}
       \centering
       \includegraphics[width=1\linewidth]{graphics/prml/{Figure1.4d}.pdf}
   \end{subfigure}
      \caption{Plots of polynomials for the model in (\ref{eq:lin-reg-model-1}) having various orders $M$, shown as red curves \cite{Bishop:2006:PRM:1162264}.}
      \label{fig:lin-reg-overfit}
\end{figure}

As we increase the number of parameters, our model becomes more flexible and then our error function approximates of zero for the training data. But when compared to the test data, the error increases. This is known as \textit{over-fitting} as seen in Figure \ref{fig:lin-reg-overfit} for $M=9$.

\subsection{Regularized Linear Regression}

\begin{table}[htpb]
   \centering
   \begin{tabular}{l| l l l l}
     & $M=0$ & $M=1$ & $M=6$ & $M=9$ \\
   \hline
   $\mathbf{w}_0^*$ & 0.19 & 0.82   & 0.31   & 0.35\\
   $\mathbf{w}_1^*$ &      & -1.27  & 7.99   & 232.37\\
   $\mathbf{w}_2^*$ &      &        & -25.43 & -5321.83\\
   $\mathbf{w}_3^*$ &      &        & 17.37  & 48568.31\\
   $\mathbf{w}_4^*$ &      &        &        & -231639.30\\
   $\mathbf{w}_5^*$ &      &        &        & 640042.26\\
   $\mathbf{w}_6^*$ &      &        &        & -1061800.52\\
   $\mathbf{w}_7^*$ &      &        &        & 1042400.18\\
   $\mathbf{w}_8^*$ &      &        &        & -557682.99\\
   $\mathbf{w}_9^*$ &      &        &        & 125201.43\\
   \end{tabular}
   \caption{Table of the coefficients $\mathbf{w} ^*$ for polynomials of various order. Observe how the typical magnitude of the coefficients increases dramatically as the order of the polynomial increases \cite{Bishop:2006:PRM:1162264}.}
   \label{tab:reg-lin-reg-weights}
   \end{table}

An approach to controls the over-fitting problem is to deal with the flexibility of the model. In the Table \ref{tab:reg-lin-reg-weights} we can note that, for a larger number of parameters, the derivative of the function $y$ takes larger constants. Then to control the over-fitting, we can be done by controlling the norm of $\mathbf{w}^*$ as the number of parameters increases. By (\ref{eq:lin-reg-error-function-1}) we can add the penalty term $||\mathbf{w}||^2 $ scaled by the factor $\lambda / 2$, then


\begin{subequations}
   \begin{equation}
      \label{eq:lin-reg-error-function-regularized}
      E(\mathbf{w}) = \frac{1}{2} \sum_{n=1}^N \left\{ y(x_i,\mathbf{w})-y_i \right\}^2 + \frac{\lambda}{2} ||\mathbf{w}||^2
   \end{equation}
   \begin{equation}
      \label{eq:lin-reg-error-function-regularized-matrix-form}
      \mathbf{w}^{*} = (\Phi^\top \Phi + \lambda \mathbf{I})^{-1}\Phi^\top \mathbf{t} 
   \end{equation}
\end{subequations}

whats means that our error increases as the norm of $\mathbf{w}$ grows, ans we can control this increasing by adjusting the term $\lambda$ as we want. By this we see in \autoref{fig:reg-lin-reg-overfit-lambda} that the regularization acts like a \textit{smoothing} factor over the function.


% This allow us to increase the number of parameters trying to control the over-fitting. More, add parameters will allows us to capture different aspects of the data set, what we will see later.

\begin{figure}[htpb]
   \centering
   \begin{subfigure}[htpb]{0.45\linewidth}
       \centering
       \includegraphics[width=1\linewidth]{graphics/prml/{Figure1.7a}.pdf}
   \end{subfigure}
   \hfill
   \begin{subfigure}[htpb]{0.45\linewidth}
       \centering
       \includegraphics[width=1\linewidth]{graphics/prml/{Figure1.7b}.pdf}       
   \end{subfigure}
      \caption{Plots of $M = 9$ polynomials fitted to the data set using the regularized error function (\ref{eq:lin-reg-error-function-regularized}) for two values of the regularization parameter $\lambda$ corresponding to $\ln \lambda = -18$ and $\ln \lambda = 0$. The case of no regularizer, i.e., $\lambda = 0$, corresponding to $\ln \lambda = -\infty$, is shown at the bottom right of \autoref{fig:lin-reg-overfit} \cite{Bishop:2006:PRM:1162264}.}
      \label{fig:reg-lin-reg-overfit-lambda}
\end{figure}

\section{Bayesian Linear Regression}

\subsection{A Bayesian view of Linear Regression}

Until now, we see the curve fitting problem in terms of the minimization of the error function. Then we will see the same by a probabilistic perspective gaining some insights into error minimization and regularization, leading us to a full Bayesian treatment.

\begin{equation}
   \label{eq:bay-reg-bayes-rule}
   p(\mathbf{w} | \mathcal{D})=\frac{p(\mathcal{D} | \mathbf{w}) p(\mathbf{w})}{p(\mathcal{D})}
\end{equation}

We can use the Bayes' theorem (\ref{eq:bay-reg-bayes-rule}) to convert a \textit{prior} probability into a \textit{posterior} probability at the light of some evidence. We can make assumptions about quantities such as the parameters $\mathbf{w}$ in the form of a prior distribution $p(\mathbf{w})$. The observation of the data $\mathcal{D}$ and what its implies in the parameters is expressed as a conditional probability $p(\mathcal{D}|\mathbf{w})$. Then we can evaluate the uncertainty about $\mathbf{w}$ after observed the data $\mathcal{D}$ as a posterior probability $p(\mathbf{w}|\mathcal{D})$.

The quantity $p(\mathcal{D}|\mathbf{w})$ expresses how probable the observed data $\mathcal{D}$ is for different settings of $\mathbf{w}$. Then, not being necessarily a probability distribution, but a function over the parameters \cite{degroot2012probability}, its integral with respect to $\mathbf{w}$ could not be equal one, then to normalize the equation with respect to the left-side there's a term $p(\mathcal{D})$. This distribution is called \textit{likelihood function}.

Integrating the both sides with respect to $\mathbf{w}$, we obtain the denominator, then considering that integrating a probability distribution over itself is equal to one, we have

\begin{equation}
   \label{eq:bay-lin-bayes-theorem}
   p(\mathcal{D})=\int p(\mathcal{D} | \mathbf{w}) p(\mathbf{w}) \mathrm{d} \mathbf{w}
\end{equation}

\subsection{Bayesian curve fitting}


Let's consider the same data set $\mathcal{D}$ presented before, but now we have some uncertainty over the value of the measured value $t$. This uncertainty can be represented as a probability distribution function $p$, in this particular case a Gaussian distribution, with a mean equal to the model $y(x,\mathbf{w})$. Thus we have

\begin{equation}
   \label{eq:bay-cur-fit-1}
   p(t | x, \mathbf{w}, \beta)=\mathcal{N}\left(t | y(x, \mathbf{w}), \beta^{-1}\right)
\end{equation}


\begin{figure}[htpb]
   \centering
   \includegraphics{graphics/prml/{Figure1.16}}
   \caption{Schematic illustration of a Gaussian conditional distribution for $t$ given $x$ given by  (\ref{eq:bay-cur-fit-1}), in which the mean is given by the polynomial function y(x, w), and the precision is given by the parameter $\beta$, which is related to the variance $\beta^{-1}=\sigma^2$ \cite{Bishop:2006:PRM:1162264}.}
   % \label{<label>}
\end{figure}

Where $\beta$ is the variance of the distribution. Note that a large $\beta$ will give us more imprecision about the measured value $t$, then we can call it of \textit{precision parameter}, i.e. how much certain we are about $t$. As we done in linear regression, we are trying to obtain the parameters for the model. In other words, given a value $t$, we trying to obtain the \textit{mean} and the \textit{variance} which maximize the probability of the measured value. Assuming the data set being independent and identically distributed, the joint probability of the whole data set will be

\begin{equation}
   p(\mathbf{t} | \mathbf{x}, \mathbf{w}, \beta)= \prod_{i=0}^{N-1} \mathcal{N}\left(t_i | y(x_i, \mathbf{w}), \beta^{-1}\right)
\end{equation}

When viewed as function of $y(x_i, \mathbf{w})$, the model, and $\beta^{-1}$, this is the likelihood function for the Gaussian. The parameters of the distribution can be determined by maximizing the likelihood function. It is convenient to maximize the log of the likelihood function, or minimize the negative log whats is equivalent, this implies that the maximization of the log of the function is equivalent to the maximization of the function itself, because the logarithm is a monotonically increasing of its argument. This helps the mathematical analysis and helps numerically because the small probabilities can easily underflow the numerical precision of the computer. Then\footnote[2]{Consider the Gaussian distribution as $\mathcal{N}\left(x | \mu, \sigma^{2}\right)=\left(2 \pi \sigma^{2}\right)^{-1 / 2} \exp \left\{-\frac{1}{2 \sigma^{2}}(x-\mu)^{2}\right\}$}

\begin{equation}
   \label{eq:bay-reg-ml-function}
   \ln p(\mathbf{t} | \mathbf{x}, \mathbf{w}, \beta)=-\frac{\beta}{2} \sum_{i=1}^{N}\left\{y\left(x_{i}, \mathbf{w}\right)-t_{i}\right\}^{2}+\frac{N}{2} \ln \beta-\frac{N}{2} \ln (2 \pi)
\end{equation}

The maximization of (\ref{eq:bay-reg-ml-function}) taking the derivative with respect to $\mathbf{w}$ will lead us back to 
the same of the minimization of (\ref{eq:lin-reg-error-function-1}), the error function of the linear regression. Here, just by notation, we will call the resulting parameters of the maximization of $\mathbf{w}_{\text{ML}}$, what it is called \textit{maximum likelihood}.

We can determine the precision parameter using the maximum likelihood by taking the derivative with respect to $\beta$ of (\ref{eq:bay-reg-ml-function}), what gives

\begin{equation}
   \frac{1}{\beta_{\mathrm{ML}}}=\frac{1}{N} \sum_{i=0}^{N-1}\left\{y\left(x_{i}, \mathbf{w}_{\mathrm{ML}}\right)-t_{i}\right\}^{2}
\end{equation}

Now we have a probabilistic view of the regression and then we can make predictions for new values of $x$, given that our model is capable of learn the parameters. And not just one collection of them, but a distribution probability.

In other words, after find the maximum likelihood parameters $\mathbf{w}_\text{ML}$ and $\beta_\text{ML}$, we have the parameters distribution by

\begin{equation}
   p\left(t | x, \mathbf{w}_{\mathrm{ML}}, \beta_{\mathrm{ML}}\right)=\mathcal{N}\left(t | y\left(x, \mathbf{w}_{\mathrm{ML}}\right), \beta_{\mathrm{ML}}^{-1}\right)
\end{equation}

Aiming to apply a "more Bayesian" approach, we not have yet a prior distribution to make the inference using the Bayes' rule. We can now introduce here the probability distribution over the parameters $p(\mathbf{w})$ as presented in the Section 3.1. The choice is arbitrary, but for this particular case we will consider

\begin{equation}
   \label{eq:bay-lin-reg-prior-dist}
   p(\mathbf{w} | \alpha)=\mathcal{N}\left(\mathbf{w} | \mathbf{0}, \alpha^{-1} \mathbf{I}\right)=\left(\frac{\alpha}{2 \pi}\right)^{(M+1) / 2} \exp \left\{-\frac{\alpha}{2} \mathbf{w}^\top \mathbf{w}\right\}
\end{equation}

where $\alpha$ is the variance, or precision parameter, of the distribution and $M+1$ is the number of parameters of the model, i.e. the length of $\mathbf{w}$. We call \textit{hyperparameters} the variables such $\alpha$ who control the model parameters distribution. And now we have by the Bayes' theorem considering that our \textit{posterior} distribution is proportional to the product between the \textit{likelihood function} and the assumed \textit{prior}, as seen in (\ref{eq:bay-lin-bayes-theorem}) then, assuming the observation of the whole data set

\begin{equation}
   p(\mathbf{w} | \mathbf{x}, \mathbf{t}, \alpha, \beta) \propto p(\mathbf{t} | \mathbf{x}, \mathbf{w}, \beta) p(\mathbf{w} | \alpha)
\end{equation}

As done before, we maximize the posterior probability, i.e. find the most probable value given the data by the term $p(\mathbf{w} | \mathbf{x}, \mathbf{t}) $ aside of the distribution parameters. This will result a particular choice of $\mathbf{w}$. We call this approach of \textit{maximum posterior}, or MAP. Then taking the negative logarithm
%
\begin{equation}
   \ln p(\mathbf{w} | \mathbf{x}, \mathbf{t}, \alpha, \beta) \propto \ln p(\mathbf{t} | \mathbf{x}, \mathbf{w}, \beta) + \ln p(\mathbf{w} | \alpha)
\end{equation}
%
Then we substitute the probability distributions founded before. Note that the first term in the right side is the error function founded in (\ref{eq:bay-reg-ml-function}). Then, the terms which the minimization depends of $\mathbf{w}$ are

\begin{equation}
    \frac{\beta}{2} \sum_{i=1}^{N}\left\{y\left(x_{i}, \mathbf{w}\right)-t_{i}\right\}^{2} + \frac{\alpha}{2} \mathbf{w}^\top \mathbf{w}
\end{equation}

And we can note the similarity with the regularized linear regression in (\ref{eq:lin-reg-error-function-regularized}) aside of the term $\lambda$, what can be founded by $\lambda=\alpha / \beta$. It's important to note that even called maximum posterior, here it was presented a minimization in terms of the negative logarithm, but this equals to the maximization of the positive logarithm. The signal was chosen just for similarity with the error function.

\subsection{Bayesian inference}

The similarity between the maximization and the Bayesian approach mentioned before shows that the second comprise even a model training such as the classical regression, as also the control of the over fitting by the regularization. But to say that our model is in fact Bayesian, we might obtain not just a single value, as in MAP, but its distribution. This requires the application the fully Bayes' theorem as (\ref{eq:bay-reg-bayes-rule}). Then, we have

\begin{equation}
   \overbrace{p(\mathbf{w}|\mathbf{t})}^{posterior} = \frac{p(\mathbf{t}|\mathbf{w}) p(\mathbf{w})}{p(\mathbf{t})} = \frac{\overbrace{p(\mathbf{t}|\mathbf{w})}^{\text{\textit{likelihood}}} \overbrace{p(\mathbf{w})}^{\text{\textit{prior}}}}{\underbrace{\int p(\mathbf{t}|\mathbf{w}) p(\mathbf{w}) d\mathbf{w}}_{\text{\textit{marginal distribution}}}}
\end{equation}

This is called \textit{Bayesian inference}. If we assume that all distributions which we are working are Gaussian, the posterior distribution has closed form. To do that, we make use of the closure under linear transformations, or \textit{affine transformations}, for the Gaussian. For that, we will make use of the corollary below, which theorems are proven in the \autoref{subsec:app-par-gau}.

\begin{corollary}

   Being $\mathbf{x}_b$ conditioned on $\mathbf{x}_a$ and Gaussian distributed as

   \begin{equation}
     p\left(\mathbf{x}_{a}\right)=\mathcal{N}\left(\mathbf{x}_{a} | \boldsymbol{\mu}_{a}, \boldsymbol{\Sigma}_{a}\right), \quad p\left(\mathbf{x}_{b} | \mathbf{x}_{a}\right)=\mathcal{N}\left(\mathbf{x}_{b} | \mathbf{M} \mathbf{x}_{a}+\mathbf{d}, \boldsymbol{\Sigma}_{b | a}\right)
   \end{equation}

   with $\boldsymbol{\mu}_{b} =\mathbf{M} \boldsymbol{\mu}_{a}+\mathbf{d}$, $\boldsymbol{\Sigma}_{b} =\boldsymbol{\Sigma}_{b | a}+\mathbf{M} \boldsymbol{\Sigma}_{a} \mathbf{M}^\top$, $\mathbf{M}$ a constant matrix and $\mathbf{d}$ a constant vector, both with the appropriate dimensions. Then conditional distribution $p(\mathbf{x}_a|\mathbf{x}_b)$ is given by

   \begin{subequations}
   
   \begin{align}
     p\left(\mathbf{x}_{a} | \mathbf{x}_{b}\right)&=\mathcal{N}\left(\mathbf{x}_{a} | \boldsymbol{\mu}_{a | b}, \boldsymbol{\Sigma}_{a | b}\right)
   \end{align}
   with
   \begin{align}
       \boldsymbol{\mu}_{a | b}&=\boldsymbol{\Sigma}_{a | b}\left(\mathbf{M}^\top \boldsymbol{\Sigma}_{b | a}^{-1}\left(\mathbf{x}_{b}-\mathbf{d}\right)+\boldsymbol{\Sigma}_{a}^{-1} \boldsymbol{\mu}_{a}\right) \\ \boldsymbol{\Sigma}_{a | b}&=\left(\boldsymbol{\Sigma}_{a}^{-1}+\mathbf{M}^\top \boldsymbol{\Sigma}_{b | a}^{-1} \mathbf{M}\right)^{-1}.
   \end{align}
 \end{subequations}
 \end{corollary}

 \begin{figure}[htpb]
   \centering
   \includegraphics[width=.95\linewidth]{graphics/prml/{Figure3.7}}
   \caption{Illustration of sequential Bayesian learning for a simple linear model of the form $y(x, \mathbf{w}) = w_0 + w_1 x$. The hyperparameters $\alpha$ and $\beta$ are assumed as $2$ and $25$, respectively, just by example \cite{Bishop:2006:PRM:1162264}.}
   % \label{<label>}
\end{figure}

 Assuming no deviation in the mean, $\mathbf{d} = \mathbf{0}$, and the linear transformation being our design matrix, $\mathbf{M} = \Phi$, we obtain that

 \begin{equation}
   \label{eq:bay-inf-posterior}
    p(\mathbf{w}|\mathbf{t},\alpha,\beta) = \mathcal{N} \left( \mathbf{w} | \boldsymbol{\mu}_{\mathbf{w}|\mathbf{t}} , \boldsymbol{\Sigma}_{\mathbf{w}|\mathbf{t}}\right)
 \end{equation}

 being

 \begin{equation}
   \boldsymbol{\mu}_{\mathbf{w}|\mathbf{t}}=\boldsymbol{\Sigma}_{\mathbf{w}|\mathbf{t}}\left(\beta \Phi^\top \mathbf{t}+\boldsymbol{\Sigma}_{\mathbf{w}}^{-1} \boldsymbol{\mu}_{\mathbf{w}}\right), \quad \boldsymbol{\Sigma}_{\mathbf{w}|\mathbf{t}}=\left(\boldsymbol{\Sigma}_{\mathbf{w}}^{-1}+ \beta \Phi^\top \Phi\right)^{-1}
 \end{equation}

 assumed the prior distribution defined in (\ref{eq:bay-lin-reg-prior-dist}) and the precision matrix $\boldsymbol{\Sigma}_{\mathbf{t}|\mathbf{w}} = \beta^{-1}\mathbf{I}$. Then we have defined

 \begin{equation}
    \boldsymbol{\mu}_\mathbf{w}=\mathbf{0}, \quad \boldsymbol{\Sigma}_\mathbf{w} = \alpha^{-1} \mathbf{I}
 \end{equation}

\subsection{Predictive distribution}

In practice, some times it is more valuable the information about $t$ itself than its parameters $\mathbf{w}$. We can make this by evaluating the predictions of $t$ for the new values of $x$ by

\begin{equation}
   p(t | \mathbf{t}, \alpha, \beta)=\int p(t | \mathbf{w}, \beta) p(\mathbf{w} | \mathbf{t}, \alpha, \beta) \mathrm{d} \mathbf{w}
\end{equation}

what is called \textit{predictive distribution}. The distributions under integration were defined in (\ref{eq:bay-cur-fit-1}) and (\ref{eq:bay-inf-posterior}). Then we use the \textit{marginalization} defined in \autoref{subsec:app-par-gau} and obtain that

\begin{equation}
   p(t | \mathbf{x}, \mathbf{t}, \alpha, \beta)=\mathcal{N}\left(t | \boldsymbol{\mu}_{t}, \boldsymbol{\Sigma}_{t} \right)
\end{equation}

with $\boldsymbol{\mu}_y = \boldsymbol{\phi}(x)^\top \boldsymbol{\Sigma}_{\mathbf{w} | \mathbf{t}}$ and $\boldsymbol{\Sigma}_y = \beta^{-1} + \boldsymbol{\phi}(x)^\top \boldsymbol{\Sigma}_{\mathbf{w} | \mathbf{t}}\boldsymbol{\phi}(x)$. Note that we are considering the linear model as $y(x,\mathbf{w}) = \boldsymbol{\phi}(x)^\top \mathbf{w}$, i.e. the affine transformation $\mathbf{M}$ here is $\boldsymbol{\phi}(x)^\top$.

An alternative formulation \cite{Rasmussen:2005:GPM:1162254} is

\begin{equation}
\begin{aligned}
      p(t | \mathbf{x}, \mathbf{t}, \alpha, \beta) = \mathcal{N} & \left(  \boldsymbol{\phi}(x)^{\top} \Sigma_\mathbf{w} \Phi\left(K+\beta I\right)^{-1} \mathbf{t} \right., \\ &\left.\boldsymbol{\phi}(x)^{\top} \Sigma_\mathbf{w} \boldsymbol{\phi}(x)-\boldsymbol{\phi}(x)^{\top} \Sigma_\mathbf{w} \Phi\left(K+\beta I\right)^{-1} \Phi^{\top} \Sigma_\mathbf{w} \boldsymbol{\phi}(x)\right)
\end{aligned}
\end{equation}

with $K=\Phi^{\top} \boldsymbol{\Sigma}_\mathbf{w} \Phi$.

\subsection{Kernels}

In the linear regression, in particular particular, we fit the data using a polynomial function of the form
%
\begin{equation}
      f(x, \mathbf{w})=w_{0}+w_{1} x+w_{2} x^{2}+\ldots+w_{M} x^{M}=\sum_{j=0}^{M} w_{j} x^{j}
      \label{eq:intro-poly}
\end{equation}
%
where $M$ is the \textit{order} of the polynomial, and $x^j$ denotes $x$ raised to the power of $j$. The polynomial coefficients $w_0, \dots , w_M$ are collectively denoted by the vector $\mathbf{w}$. Note that, although the polynomial function $f(x, \mathbf{w})$ is a nonlinear function of $x$, it is a linear function of the coefficients $w$. Functions, such as the polynomial, which are linear in the unknown parameters have important properties and are called \textit{linear models}. In general, we could write this weighted sum with any other function. In other words, we can put this in terms of $\phi_n(x)=x^n$, where $\phi$ could be other \textit{basis function}, e.g. we could have different functions $f$ for different basis functions.
%
\begin{align*}
      f(x,\mathbf{w}) &= w_0 \phi_0(x) +w_1 \phi_1(x) +w_2 \phi_2(x)  + ... + w_{M-1} \phi_{M-1}(x); \\
                        &= w_0 \exp\left\{ - \frac{(x-\mu_0)^2}{2\sigma^2}\right\} + w_1  \exp\left\{ - \frac{(x-\mu_1)^2}{2\sigma^2}\right\} + \\ & ... + w_{M-1} \exp\left\{ - \frac{(x-\mu_{M-1})^2}{2\sigma^2}\right\}; \\
                        &= w_0 \sin(0 \cdot x) + w_1 \cos(1 \cdot x) + \\ &... + w_{M-2} \sin((M-2) \cdot x) + w_{M-1} \cos((M-1) \cdot x); \\
                        &=\sum_{j=0}^{M-1} w_{j} \phi_{j}(x);
\end{align*}
%

With this, our linear model is able to capture different aspects of the data set, as periodicity, exponential increasing, roughness etc. When the number of basis functions tends to infinite these models become \textit{kernels} \cite{mackay1998introduction}. It's important to note that these linear models are needed to define its basis functions before the training data set is observed.

Notice that in the previous section, we are using transformations always of the type $\Phi^{\top} \boldsymbol{\Sigma}_\mathbf{w} \Phi$, $\boldsymbol{\phi}(x)^{\top} \boldsymbol{\Sigma}_\mathbf{w} \Phi$, or $\boldsymbol{\phi}(x)^{\top} \boldsymbol{\Sigma}_\mathbf{w} \boldsymbol{\phi}(x)$. Then we can generalize the form $\boldsymbol{\phi}(\mathbf{x})^\top \boldsymbol{\Sigma}_\mathbf{w}\boldsymbol{\phi}(\mathbf{x'})$ where $\mathbf{x}$ and $\mathbf{x'}$ are in either the training or the test sets. We define this form as $k(\mathbf{x},\mathbf{x'})$, where $k(\cdot,\cdot)$ is called \textit{covariance function}, or \textit{kernel}. Further insight into the role of the equivalent kernel can be obtained by considering the covariance between $y(\mathbf{x})$ and $y(\mathbf{x'})$, which is given by

\begin{equation}
   \begin{aligned}
      \mathrm{cov}\left\{ y(\mathbf{x}), y(\mathbf{x'})\right\} &=\mathrm{cov} \left\{ \boldsymbol{\phi}(\mathbf{x})^\top \mathbf{w}, \mathbf{w}^\top \boldsymbol{\phi}(\mathbf{x}) \right\} \\
       &= \boldsymbol{\phi}(\mathbf{x})^\top \boldsymbol{\Sigma}_\mathbf{w}\boldsymbol{\phi}(\mathbf{x}) = \beta^{-1}k(\mathbf{x},\mathbf{x'})
   \end{aligned}
\end{equation}

From the form of the kernel, we see that the predictive mean at nearby points will be highly correlated, whereas for more distant pairs of points the correlation will be smaller. The linearity preserves the propriety of addition, then our model accepts the junction of others functions to create a new kernel, in order to capture some aspects of the data set. 

\section{Gaussian processes}

Until now we have made the inference in the \textit{feature space}, the space where the parameters $\mathbf{w}$ are. In other words, the strategy was to train our model, obtaining the parameters probability distribution, by Bayesian inference, and then evaluating the predictive distribution with the posterior of the inference.

An alternative and equivalent way to achieve such results is to make the inference directly in the space of functions, or \textit{function space}. To this we use the \textit{Gaussian processes} to describe the distribution over the functions directly \cite{Rasmussen:2005:GPM:1162254}.

We define a Gaussian process (GP) as a collection of random variables, such that any finite number of which is normal jointly distributed. In other words, it can be thought  of as a generalization of a Gaussian distribution over a finite vector space to a function space of infinite dimension \cite{mackay1998introduction}. As the normal distribution, the GP is completely defined by its mean function $m(\mathbf{x})$ and covariance function $k(\mathbf{x},\mathbf{x'})$ of a real process $y(\mathbf{x})$, these in turn are defined as

\begin{equation}
   m(\mathbf{x}) = \mathbb{E}\left\{ y(\mathbf{x}) \right\}, \quad
   k(\mathbf{x},\mathbf{x'}) = \mathbb{E}\left\{ \left( y(\mathbf{x}) - m(\mathbf{x}) \right) \left( y(\mathbf{x'}) - m(\mathbf{x'}) \right) \right\}
\end{equation}

and finally

\begin{equation}
   y(\mathbf{x}) \sim \mathcal{GP} \left( m(\mathbf{x}), k(\mathbf{x},\mathbf{x'}) \right)
\end{equation}

Such collection definition automatically implies in the marginalization property already present for the multidimensional Gaussian distributions. For the GP this means that the observation of a larger set of variables does not change the distribution of the smaller set.

This is important to obtain our Bayesian linear regression as a GP. Being our model $y(\mathbf{x})=\boldsymbol{\phi}(\mathbf{x})^\top \mathbf{w}$ with prior $\mathcal{N}\left( \mathbf{w} | \mathbf{0}, \boldsymbol{\Sigma}_{\mathbf{w}}\right)$. We obtain for the mean and covariance functions

\begin{equation}
   \label{eq:gau-pro-mean-cov}
   \begin{aligned}
      \mathbb{E}\left\{ y(\mathbf{x}) \right\} &= \boldsymbol{\phi}(\mathbf{x})^\top \mathbb{E} \left\{ \mathbf{w} \right\} = 0,\\
      \mathbb{E}\left\{ y(\mathbf{x})y(\mathbf{x'}) \right\} &= \boldsymbol{\phi}(\mathbf{x})^\top \mathbb{E} \left\{ \mathbf{ww}^\top \right\}\boldsymbol{\phi}(\mathbf{x}) = \boldsymbol{\phi}(\mathbf{x})^\top \boldsymbol{\Sigma}_{\mathbf{w}} \boldsymbol{\phi}(\mathbf{x'})
   \end{aligned}
\end{equation}

\subsection{Prediction with Noisy Observations}

The last sections give us an insight about the construction of the learning process for the GP. Analogous to the Bayes' theorem for the Gaussian, we use the idea of the GP as a multidimensional Gaussian distribution and we can make inference with its partitions. 

First, to take the concept, we will consider the case where the observations are noise free. We will substitute the covariance matrices from the partitioned Gaussian distributions by the covariance function applied at the points of the observations $\mathbf{t}$ and the prediction $\mathbf{t_*}$, as

\begin{equation}
   \left( \begin{array}{l}{\mathbf{t}}  \\ {\mathbf{t}_*} \end{array} \right) 
   \sim
   \mathcal{N} \left( \mathbf{0}, \left[ \begin{array}{ll}{\mathbf{K}(\mathbf{x},\mathbf{x})} & {\mathbf{K}(\mathbf{x},\mathbf{x_*})} \\ {\mathbf{K}(\mathbf{x_*},\mathbf{x})} & {\mathbf{K}(\mathbf{x_*},\mathbf{x_*})}\end{array} \right] \right)
\end{equation}

where $\mathbf{K}(\cdot,\cdot)$ denotes the covariance matrices evaluated at all pairs of $N$-dimensional $\mathbf{x}$ training points and $N_*$-dimensional $\mathbf{x_*}$ test points. Then, making use of \autoref{subsec:app-par-gau}, we use the \textit{conditioning} to obtain the predictive distribution

\begin{equation}
\begin{aligned}
      p(\mathbf{y_*}|\mathbf{x_*},\mathbf{x},\mathbf{y}) = \mathcal{N} & \left( \mathbf{y_*} | \mathbf{K}(\mathbf{x_*},\mathbf{x})\mathbf{K}(\mathbf{x},\mathbf{x})^{-1}\mathbf{y}, \right. \\ & \left. \mathbf{K}(\mathbf{x_*},\mathbf{x_*})-\mathbf{K}(\mathbf{x_*},\mathbf{x})\mathbf{K}(\mathbf{x},\mathbf{x})^{-1}\mathbf{K}(\mathbf{x},\mathbf{x_*}) \right)
\end{aligned}
\end{equation}

Now assuming the noise in the observations, what is a more realistic modelling situation, we have that $t = y(\mathbf{x}) + \varepsilon$, being $\varepsilon$ the Gaussian noise with variance $\beta^{-1}$. Then we have that $\mathrm{cov}(\mathbf{t})=\mathbf{K}(\mathbf{x},\mathbf{x}) + \beta^{-1}\mathbf{I}$.

Deriving the conditional distribution corresponding we arrive at the key predictive equations for Gaussian process regression

\begin{equation}
\begin{aligned} p(\mathbf{y}_{*} | \mathbf{x}, \mathbf{t},\mathbf {x_{*}}) &= \mathcal{N}\left(\overline{\mathbf{y}}_{*}, \mathrm{cov}\left(\mathbf{y}_{*}\right)\right), \text { where } \\ \overline{\mathbf{y}}_{*} & \triangleq \mathbb{E}\left\{ \mathbf{y}_{*} | \mathbf{x}, \mathbf{t}, \mathbf{x_{*}}\right\}=\mathbf{K}\left(\mathbf{x_{*}}, \mathbf{x}\right)\left(\mathbf{K}(\mathbf{x}, \mathbf{x})+\beta^{-1} \mathbf{I}\right)^{-1} \mathbf{t} \\ \mathrm{cov}\left(\mathbf{y}_{*}\right) &=\mathbf{K}\left(\mathbf{x_{*}}, \mathbf{x_{*}}\right)-\mathbf{K}\left(\mathbf{x_{*}}, \mathbf{x}\right)\left(\mathbf{K}(\mathbf{x}, \mathbf{x})+\beta^{-1} \mathbf{I}\right)^{-1} \mathbf{K}\left(\mathbf{x}, \mathbf{x_{*}}\right) \end{aligned}
\end{equation}

\begin{figure}[htpb]
   \centering
   \includegraphics{graphics/prml/{Figure6.8}}
   \caption{Illustration of Gaussian process regression applied to the sinusoidal data set in which the three right-most data points have been omitted. The green curve shows the sinusoidal function from which the data points, shown in blue, are obtained by sampling and addition of Gaussian noise. The red line shows the mean of the Gaussian process predictive distribution, and the shaded region corresponds to plus and minus two standard deviations. Notice how the uncertainty increases in the region to the right of the data points \cite{Bishop:2006:PRM:1162264}.}
   % \label{<label>}
\end{figure}

\section{Hyperparameters}

As in the distributions, the kernels have its parameters to be set before the model training. Such parameters, called hyperparameters can assume a distribution over itself, and with this, we can learn the parameters to the data just as we've been doing for the inference. In general, the posterior distribution for the hyperparameters are intractable, then we cal use MAP to choose at least one value (the most probable one), or evaluate the integral numerically.

\newpage
\section*{\LARGE{Appendix}}

\begin{appendices}
  \vspace{4em}
  \section{Derivations}
  
  \subsection{Matrix Form}
  \label{subsec:app-matrix-form}
  
  Be the linear model $f(x,\mathbf{w}) = \mathbf{w}^\top \boldsymbol{\phi}(x)$. Suppose $\Phi = \left[ \boldsymbol{\phi}(x_1),\dots,\boldsymbol{\phi}(x_N) \right]^\top$, then $\Phi$ will be of the form


\begin{equation}
    \Phi =
    \begin{bmatrix}
      \phi_0(x_0) & \dots & \phi_{M-1}(x_{0}) \\
      \vdots & \ddots & \vdots \\
      \phi_{0}(x_{N-1}) & \dots & \phi_{M-1}(x_{N-1})
    \end{bmatrix}
\end{equation}

called \textit{design matrix}. Then the model turns to $\mathbf{f} = \Phi \mathbf{w}$. This will lead us to the matrix form for the quadratic error function

\begin{align*}
  E(\mathbf{w}) &= \frac{1}{2}(\mathbf{f} - \mathbf{y})^\top (\mathbf{f} - \mathbf{y}) \\
                &= \frac{1}{2}(\Phi \mathbf{w} - \mathbf{y})^\top (\Phi \mathbf{w} - \mathbf{y}) \\
                &= \frac{1}{2}(\mathbf{w}^\top \Phi^\top \Phi \mathbf{w}  - \mathbf{y}^\top \Phi \mathbf{w} - \mathbf{w}^\top \Phi^\top \mathbf{y} + \mathbf{y}^\top \mathbf{y}) \nonumber \\
\end{align*}

Observe that even in the matrix form, the error function remains scalar, which implies that $\mathbf{y}^\top \Phi \mathbf{w} = \mathbf{w}^\top \Phi^\top \mathbf{y}$ by the transpose of the product rule. Then

\begin{align*}
  E(\mathbf{w}) &= \frac{1}{2}(\mathbf{w}^\top \Phi^\top \Phi \mathbf{w}  - 2\mathbf{y}^\top \Phi \mathbf{w} + \mathbf{y}^\top \mathbf{y}) \nonumber \\
\end{align*}

Then we proceed by the minimization by $\frac{\partial E}{\partial \mathbf{w}} = 0$

\begin{align}
  0 &= \frac{1}{2}(2\mathbf{w}^\top \Phi^\top \Phi - 2\mathbf{y}^\top \Phi) \footnotemark[2] \nonumber \\
  \mathbf{w}^{*\top} &= \mathbf{y}^\top \Phi (\Phi^\top \Phi) ^{-1} \nonumber \\
  \mathbf{w}^{*} &= (\Phi^\top \Phi)^{-1}\Phi^\top \mathbf{y} 
\end{align}\footnotetext[2]{Using two facts. First, if $\alpha=\mathbf{x}^{\mathrm{T}} \mathbf{A} \mathbf{x}$, then $\frac{\partial \alpha}{\partial \mathbf{x}}=2 \mathbf{x}^{\top} \mathbf{A}$, being $\alpha$ scalar. Second, if $\alpha=\mathbf{y}^{\mathrm{T}} \mathbf{A} \mathbf{x}$, then $\frac{\partial \alpha}{\partial \mathbf{x}}=\mathbf{y}^{\mathrm{T}} \mathbf{A}$. For both, $\mathbf{A}$ is independent of $\mathbf{x}$ and $\mathbf{y}$ \cite{graybill1983matrices}.}

For the regularized linear regression, we do $\frac{\lambda}{2} ||\mathbf{w}||^2 = \mathbf{w}^\top \mathbf{w}$, then

\begin{align*}
  E(\mathbf{w}) &= \frac{1}{2}(\mathbf{f} - \mathbf{y})^\top (\mathbf{f} - \mathbf{y}) + \frac{\lambda}{2} \mathbf{w}^\top \mathbf{w}\\
                &= \frac{1}{2}(\mathbf{w}^\top \Phi^\top \Phi \mathbf{w}  - \mathbf{y}^\top \Phi \mathbf{w} - \mathbf{w}^\top \Phi^\top \mathbf{y} + \mathbf{y}^\top \mathbf{y}) + \frac{\lambda}{2} \mathbf{w}^\top \mathbf{w} \nonumber \\
\end{align*}

And with the minimization we do $\frac{\partial E}{\partial \mathbf{w}} = 0$, then

\begin{align}
  0 &= \mathbf{w}^\top \Phi^\top \Phi - \mathbf{y}^\top \Phi + \lambda \mathbf{w}^\top\nonumber \\
  \mathbf{w}^{*\top} &= \mathbf{y}^\top \Phi (\Phi^\top \Phi + \lambda \mathbf{I}) ^{-1} \nonumber \\
  \mathbf{w}^{*} &= (\Phi^\top \Phi + \lambda \mathbf{I})^{-1}\Phi^\top \mathbf{y} 
\end{align}

where $\mathbf{I}$ is the identity matrix.

\vspace{4em}

  \section{Bayes' theorem for Gaussian variables\cite{schon_lindsten}}
  \label{subsec:app-par-gau}

  \subsection{Partitioned Gaussian distributions}
  Be $\mathbf{x}$ a n-dimensional vector with a Gaussian distribution $\mathcal{N}\left( \mathbf{x} | \boldsymbol{\mu}, \boldsymbol{\Sigma} \right)$, then the partitioned will be
  
  \begin{equation}
    \mathbf{x}=
    \begin{pmatrix}
    \mathbf{x}_a \\  
    \mathbf{x}_b 
    \end{pmatrix}
    ,\quad 
    \boldsymbol{\mu}=
    \begin{pmatrix}
      \boldsymbol{\mu}_a \\
      \boldsymbol{\mu}_b
    \end{pmatrix}
    ,\quad 
    \boldsymbol{\Sigma}=
    \begin{pmatrix}
      \boldsymbol{\Sigma}_{aa} & \boldsymbol{\Sigma}_{ab}  \\
      \boldsymbol{\Sigma}_{ba} & \boldsymbol{\Sigma}_{bb}
    \end{pmatrix}
    .
  \end{equation}

  Preserved the symmetry $\boldsymbol{\Sigma}^\top = \boldsymbol{\Sigma}$, we say the covariance matrix is positive definite. And be the multivariate Gaussian

  \begin{equation}
    \label{eq:app-par-gau-multivariate-gaussian}
    \mathcal{N}(\mathbf{x} | \boldsymbol{\mu}, \mathbf{\Sigma})=\frac{1}{(2 \pi)^{n / 2}} \frac{1}{ \left( \det \mathbf{\Sigma} \right) ^{1 / 2}} \exp \left\{-\frac{1}{2}(\mathbf{x}-\boldsymbol{\mu})^{\mathrm{T}} \boldsymbol{\Sigma}^{-1}(\mathbf{x}-\boldsymbol{\mu})\right\}
  \end{equation}

  We define too, just for convenience of work, the precision matrix $\boldsymbol{\Lambda}$ by

  \begin{equation}
    \boldsymbol{\Lambda} = 
    \begin{pmatrix}
      \boldsymbol{\Lambda}_{aa} & \boldsymbol{\Lambda}_{ab}  \\
      \boldsymbol{\Lambda}_{ba} & \boldsymbol{\Lambda}_{bb}
    \end{pmatrix} 
    \equiv \boldsymbol{\Sigma}^{-1}
  \end{equation}

  assuming all matrices have inverses.

  \begin{theorem}[Marginalization]
    \label{theorem:app-gau-margin}
    Being the random vector $\mathbf{x}$ and its partitioned as above, the marginal density $p(\mathbf{x}_a)$ is given by
    \begin{displaymath}
      p(\mathbf{x}_a) = \mathcal{N}\left( \mathbf{x}_a | \boldsymbol{\mu}_a, \boldsymbol{\Sigma}_{aa} \right)
    \end{displaymath}
  \end{theorem}

  \begin{proof}
    The marginal density $p(\mathbf{x}_a)$ is obtained by integrating the joint density $p(\mathbf{x})=p\left(\mathbf{x}_{a}, \mathbf{x}_{b}\right)$ with relation to $\mathbf{x}_b$
    \begin{equation}
      p\left(\mathbf{x}_{a}\right)=\int p\left(\mathbf{x}_{a}, \mathbf{x}_{b}\right) \mathrm{d} \mathbf{x}_{b}
    \end{equation}

    Then we expand the exponential argument of (\ref{eq:app-par-gau-multivariate-gaussian}) for the partitioned Gaussian

    \begin{equation}
      -\frac{1}{2}
      \left( \begin{pmatrix}
        \mathbf{x}_a \\  
        \mathbf{x}_b 
        \end{pmatrix}
        -\begin{pmatrix}
          \boldsymbol{\mu}_a \\
          \boldsymbol{\mu}_b
        \end{pmatrix}
        \right)^\top
        \begin{pmatrix}
          \boldsymbol{\Lambda}_{aa} & \boldsymbol{\Lambda}_{ab}  \\
          \boldsymbol{\Lambda}_{ba} & \boldsymbol{\Lambda}_{bb}
        \end{pmatrix} 
        \left( \begin{pmatrix}
          \mathbf{x}_a \\  
          \mathbf{x}_b 
          \end{pmatrix}
          -\begin{pmatrix}
            \boldsymbol{\mu}_a \\
            \boldsymbol{\mu}_b
          \end{pmatrix}
          \right)
    \end{equation}
    
    What implies

    \begin{equation}
    \begin{aligned} &-\frac{1}{2}\left(\mathbf{x}_{a}-\boldsymbol{\mu}_{a}\right)^\top \boldsymbol{\Lambda}_{a a}\left(\mathbf{x}_{a}-\boldsymbol{\mu}_{a}\right)-\frac{1}{2}\left(\mathbf{x}_{a}-\boldsymbol{\mu}_{a}\right)^\top \boldsymbol{\Lambda}_{a b}\left(\mathbf{x}_{b}-\boldsymbol{\mu}_{b}\right) \\ &-\frac{1}{2}\left(\mathbf{x}_{b}-\boldsymbol{\mu}_{b}\right)^\top \boldsymbol{\Lambda}_{b a}\left(\mathbf{x}_{a}-\boldsymbol{\mu}_{a}\right)-\frac{1}{2}\left(\mathbf{x}_{b}-\boldsymbol{\mu}_{b}\right)^\top \boldsymbol{\Lambda}_{b b}\left(\mathbf{x}_{b}-\boldsymbol{\mu}_{b}\right) 
    \end{aligned}
    \end{equation}

    Here we make use of the \textit{Schur complement}, in which, being the partitioned matrix

    \begin{equation}
    \left(\begin{array}{cc}{\mathbf{A}} & {\mathbf{B}} \\ {\mathbf{C}} & {\mathbf{D}}\end{array}\right)^{-1}=\left(\begin{array}{cc}{\mathbf{A}^{-1}+\mathbf{A}^{-1} \mathbf{B} \mathbf{M}^{-1} \mathbf{C} \mathbf{A}^{-1}} & {-\mathbf{A}^{-1} \mathbf{B} \mathbf{M}^{-1}} \\ {-\mathbf{M}^{-1} \mathbf{C} \mathbf{A}^{-1}} & {\mathbf{M}^{-1}}\end{array}\right)
    \end{equation}

    the quantity $\mathbf{M}^{-1}$ is the \textit{Schur complement} of the left side matrix with respect to $\mathbf{D}$, defined as 

    \begin{equation}
      \label{eq:app-par-gau-schur}
      \mathbf{M}=\left(\mathbf{D}-\mathbf{C A}^{-1} \mathbf{B}\right)^{-1}  
    \end{equation}

    This will motivated the term grouping below

    \begin{equation*}
    \begin{aligned} &-\frac{1}{2}\left(\mathbf{x}_{a}-\boldsymbol{\mu}_{a}\right)^\top \boldsymbol{\Lambda}_{a a}\left(\mathbf{x}_{a}-\boldsymbol{\mu}_{a}\right)-\frac{1}{2}\left(\mathbf{x}_{a}-\boldsymbol{\mu}_{a}\right)^\top \boldsymbol{\Lambda}_{a b}\left(\mathbf{x}_{b}-\boldsymbol{\mu}_{b}\right) \\ &-\frac{1}{2}\left(\mathbf{x}_{b}-\boldsymbol{\mu}_{b}\right)^\top \boldsymbol{\Lambda}_{b a}\left(\mathbf{x}_{a}-\boldsymbol{\mu}_{a}\right)-\frac{1}{2}\left(\mathbf{x}_{b}-\boldsymbol{\mu}_{b}\right)^\top \boldsymbol{\Lambda}_{b b}\left(\mathbf{x}_{b}-\boldsymbol{\mu}_{b}\right) \\ &=-\frac{1}{2}\left(\mathbf{x}_{b}^\top \boldsymbol{\Lambda}_{b b} \mathbf{x}_{b}-2 \mathbf{x}_{b}^\top \boldsymbol{\Lambda}_{b b}\left(\boldsymbol{\mu}_{b}-\boldsymbol{\Lambda}_{b b}^{-1} \boldsymbol{\Lambda}_{b a}\left(\mathbf{x}_{a}-\boldsymbol{\mu}_{a}\right)\right)-2 \mathbf{x}_{a}^\top \boldsymbol{\Lambda}_{a b} \boldsymbol{\mu}_{b}\right.\\ & \left.  +2 \boldsymbol{\mu}_{a}^\top \boldsymbol{\Lambda}_{a b} \boldsymbol{\mu}_{b}+\boldsymbol{\mu}_{b}^\top \boldsymbol{\Lambda}_{b b} \boldsymbol{\mu}_{b}+\mathbf{x}_{a}^\top \boldsymbol{\Lambda}_{a a} \mathbf{x}_{a}-2 \mathbf{x}_{a}^\top \boldsymbol{\Lambda}_{a a} \boldsymbol{\mu}_{a}+\boldsymbol{\mu}_{a}^\top \boldsymbol{\Lambda}_{a a} \boldsymbol{\mu}_{a} \right) \end{aligned}
    \end{equation*}

    Then we complete the squares resulting independent

    \begin{equation*}
    \begin{aligned} &-\frac{1}{2}\left(\mathbf{x}_{b}-\left(\boldsymbol{\mu}_{b}-\boldsymbol{\Lambda}_{b b}^{-1} \boldsymbol{\Lambda}_{b a}\left(\mathbf{x}_{a}-\boldsymbol{\mu}_{a}\right)\right)\right)^\top \boldsymbol{\Lambda}_{b b}\left(\mathbf{x}_{b}-\left(\boldsymbol{\mu}_{b}-\boldsymbol{\Lambda}_{b b}^{-1} \boldsymbol{\Lambda}_{b a}\left(\mathbf{x}_{a}-\boldsymbol{\mu}_{a}\right)\right)\right) \\ &+\frac{1}{2}\left(\mathbf{x}_{a}^\top \boldsymbol{\Lambda}_{a b} \boldsymbol{\Lambda}_{b b}^{-1} \boldsymbol{\Lambda}_{b a} \mathbf{x}_{a}-2\mathbf{x}_{a}^\top \boldsymbol{\Lambda}_{a b} \boldsymbol{\Lambda}_{b b}^{-1} \boldsymbol{\Lambda}_{b a} \boldsymbol{\mu}_{a}+\boldsymbol{\mu}_{a} \boldsymbol{\Lambda}_{a b} \boldsymbol{\Lambda}_{b b}^{-1} \boldsymbol{\Lambda}_{b a} \boldsymbol{\mu}_{a}\right) \\ &-\frac{1}{2}\left(\mathbf{x}_{a}^\top \boldsymbol{\Lambda}_{a a} \mathbf{x}_{a}-2 \mathbf{x}_{a}^\top \boldsymbol{\Lambda}_{a a} \boldsymbol{\mu}_{a}+\boldsymbol{\mu}_{a}^\top \boldsymbol{\Lambda}_{a a} \boldsymbol{\mu}_{a}\right) \\ &=\underbrace{-\frac{1}{2}\left(\mathbf{x}_{b}-\left(\boldsymbol{\mu}_{b}-\boldsymbol{\Lambda}_{b b}^{-1} \boldsymbol{\Lambda}_{b a}\left(\mathbf{x}_{a}-\boldsymbol{\mu}_{a}\right)\right)\right)^\top \boldsymbol{\Lambda}_{b b}\left(\mathbf{x}_{b}-\left(\boldsymbol{\mu}_{b}-\boldsymbol{\Lambda}_{b b}^{-1} \boldsymbol{\Lambda}_{b a}\left(\mathbf{x}_{a}-\boldsymbol{\mu}_{a}\right)\right)\right)}_{E_1} \\ &\underbrace{-\frac{1}{2}\left(\mathbf{x}_{a}-\boldsymbol{\mu}_{a}\right)^\top\left(\boldsymbol{\Lambda}_{a a}-\boldsymbol{\Lambda}_{a b} \boldsymbol{\Lambda}_{b b}^{-1} \boldsymbol{\Lambda}_{b a}\right)\left(\mathbf{x}_{a}-\boldsymbol{\mu}_{a}\right)}_{E_2} \end{aligned}
    \end{equation*}

    Now, back to the marginalization, we have

    \begin{equation}
      \label{eq:app-par-gau-marginalization-1}
      p\left(\mathbf{x}_{a}\right)=\int \frac{1}{(2 \pi)^{n / 2}} \frac{1}{ \left( \det \mathbf{\Sigma} \right) ^{1 / 2}} \exp \left\{ E_1 \right\} \exp \left\{ E_2 \right\} d \mathbf{x}_b
    \end{equation}

    By inspection, we have that, being the integral of the density function equals one and being $n_b$ the dimension of $\mathbf{x}_b$

    \begin{equation}
      \label{eq:app-par-gau-marginalization-2}
      \int \exp \left\{ E_1 \right\} d \mathbf{x}_b = \left( 2 \pi \right)^{n_b/2} \left( \det \mathbf{\Lambda}_{bb}^{-1} \right) ^{1 / 2}
    \end{equation}

    Then, substituting (\ref{eq:app-par-gau-marginalization-1}) in (\ref{eq:app-par-gau-marginalization-2}) we have

    \begin{equation*}
      p\left(\mathbf{x}_{a}\right)=\frac{1}{(2 \pi)^{n_a / 2}} \frac{\left( \det \mathbf{\Lambda}_{bb}^{-1} \right) ^{1 / 2}}{ \left( \det \mathbf{\Sigma} \right) ^{1 / 2}} \exp \left\{ E_2 \right\}
    \end{equation*}

    We will use the determinant property from Schur complement below

    \begin{equation}
    \label{eq:app-det-schur-complement}
    \det \left(\begin{array}{cc}{\mathbf{A}} & \mathbf{B} \\ \mathbf{C} & \mathbf{D} \end{array}\right)=\det \mathbf{A} \det \left(\mathbf{D}-\mathbf{C A}^{-1} \mathbf{B}\right)
    \end{equation}

    Then, substituting the values of $\boldsymbol{\Sigma}$ we have that

    \begin{equation}
      \det \boldsymbol{\Sigma}=\det \boldsymbol{\Sigma}_{a a} \det\left(\boldsymbol{\Sigma}_{b b}-\boldsymbol{\Sigma}_{b a} \boldsymbol{\Sigma}_{a a}^{-1} \boldsymbol{\Sigma}_{a b}\right)
    \end{equation}

    Using the Schur complement

    \begin{equation}
      \det \boldsymbol{\Sigma}=\det \boldsymbol{\Sigma}_{a a} \det \boldsymbol{\Lambda}_{bb}^{-1}
    \end{equation}

    Finally, using the Schur complement again, we obtain that $\boldsymbol{\Lambda}_{a a}-\boldsymbol{\Lambda}_{a b} \boldsymbol{\Lambda}_{b b}^{-1} \boldsymbol{\Lambda}_{b a} = \boldsymbol{\Sigma}_{a a}$, what concludes the proof, substituting the result in $E_2$.
  \end{proof}

  \begin{theorem}[Conditioning]
    \label{theorem:app-gau-condit}
    Being the random vector $\mathbf{x}$ and its partitioned as above, the conditional density $p(\mathbf{x}_b | \mathbf{x}_a)$ is given by
    \begin{displaymath}
      p(\mathbf{x}_b | \mathbf{x}_a) = \mathcal{N}\left( \mathbf{x}_b | \boldsymbol{\mu}_{b|a}, \boldsymbol{\Sigma}_{b|a} \right)
    \end{displaymath}
    where $\boldsymbol{\mu}_{b|a} = \boldsymbol{\mu}_{b}-\boldsymbol{\Lambda}_{b b}^{-1} \boldsymbol{\Lambda}_{b a}\left(\mathbf{x}_{a}-\boldsymbol{\mu}_{a}\right)$ and $\boldsymbol{\Sigma}_{b|a} = \boldsymbol{\Lambda}_{bb}^{-1}$.
  \end{theorem}

  \begin{proof}
    By the product rule we have

    \begin{equation}
      p(\mathbf{x}_b|\mathbf{x}_a)=\frac{p(\mathbf{x})}{p(\mathbf{x}_a)}
    \end{equation}

    We already obtain the value for $p(\mathbf{x}_a)$, then

    \begin{equation}
      p(\mathbf{x}_b|\mathbf{x}_a)=\frac{(2 \pi )^{n_a/2}}{\left( 2 \pi \right)^{n/2}} \frac{( \det \mathbf{\Sigma}_{aa} )^{n_a/2}}{\left( \det \boldsymbol{\Sigma} \right)^{1/2}} \frac{\exp \left\{ E_1 \right\} \exp \left\{ E_2 \right\}}{\exp \left\{ E_2 \right\}}
    \end{equation}
    
    Similarly to what was done in \textit{marginalization} and using the Schur complement, the result

    \begin{equation}
      p(\mathbf{x}_b|\mathbf{x}_a)=\frac{1}{\left( 2 \pi \right)^{n_b/2}} \frac{1}{\left( \det \boldsymbol{\Lambda}_{bb} \right)^{1/2}} \exp \left\{ E_1 \right\}
    \end{equation}

    And finally, by inspection of $E_1$ we have

    \begin{subequations}
      \begin{align}
        \boldsymbol{\mu}_{b|a} &=  \boldsymbol{\mu}_{b}-\boldsymbol{\Lambda}_{b b}^{-1} \boldsymbol{\Lambda}_{b a}\left(\mathbf{x}_{a}-\boldsymbol{\mu}_{a}\right) \\
        \intertext{and analogously}
        \boldsymbol{\mu}_{a|b} &=  \boldsymbol{\mu}_{a}-\boldsymbol{\Lambda}_{a a}^{-1} \boldsymbol{\Lambda}_{a b}\left(\mathbf{x}_{b}-\boldsymbol{\mu}_{b}\right)
      \end{align}
    \end{subequations}

    for $\boldsymbol{\Lambda}_{aa}$.
  \end{proof}

  \subsection{Affine transformation}
  \label{subsec:app-affine-transf-gau}
  
  Making use of the Theorems \ref{theorem:app-gau-margin} and \ref{theorem:app-gau-condit}, we can show that the Gaussian is closed under linear transformations, i.e. an affine transformation of Gaussians results in another Gaussian.
  
  \begin{theorem}[Affine transformation]
    
    Assume $\mathbf{x}_a$ and $\mathbf{x}_b$ are Gaussian distributed and $\mathbf{x}_b$ conditioned by $\mathbf{x}_a$, as

    \begin{equation}
    p\left(\mathbf{x}_{a}\right) =\mathcal{N}\left(\mathbf{x}_{a} | \boldsymbol{\mu}_{a}, \boldsymbol{\Sigma}_{a}\right), \quad p\left(\mathbf{x}_{b} | \mathbf{x}_{a}\right) =\mathcal{N}\left(\mathbf{x}_{b} | \mathbf{M} \mathbf{x}_{a}+\mathbf{d}, \boldsymbol{\Sigma}_{b | a}\right)
    \end{equation}

    where $\mathbf{M}$ is a constant matrix and $\mathbf{d}$ a constant vector, both with the appropriate dimensions. Then the joint distribution is given by

    \begin{equation}
    p\left(\mathbf{x}_{a}, \mathbf{x}_{b}\right)=\mathcal{N}\left(\left(\begin{array}{c}{\mathbf{x}_{a}} \\ {\mathbf{x}_{b}}\end{array}\right) \left|\left(\begin{array}{c}{\boldsymbol{\mu}_{a}} \\ {\mathbf{M} \boldsymbol{\mu}_{a}+\mathbf{d}}\end{array}\right.\right), \mathbf{R}\right),
    \end{equation}

    being

    \begin{equation}
    \mathbf{R}=\left(\begin{array}{cc}{\mathbf{M}^\top \boldsymbol{\Sigma}_{b | a}^{-1} \mathbf{M}+\boldsymbol{\Sigma}_{a}^{-1}} & {-\mathbf{M}^\top \boldsymbol{\Sigma}_{b | a}^{-1}} \\ {-\boldsymbol{\Sigma}_{b | a}^{-1} \mathbf{M}} & {\boldsymbol{\Sigma}_{b | a}^{-1}}\end{array}\right)^{-1}=\left(\begin{array}{cc}{\boldsymbol{\Sigma}_{a}} & {\boldsymbol{\Sigma}_{a} \mathbf{M}^\top} \\ {\mathbf{M} \boldsymbol{\Sigma}_{a}} & {\boldsymbol{\Sigma}_{b | a}+\mathbf{M} \boldsymbol{\Sigma}_{a} \mathbf{M}^\top}\end{array}\right).
    \end{equation}
    
  \end{theorem}

  \begin{proof}
    Being the vector

    \begin{equation}
      \mathbf{x} = \left( \begin{array}{c} \mathbf{x}_a \\ \mathbf{x}_b \end{array} \right)
    \end{equation}

    and its joint distribution, from the Theorems 1 and 2,

    \begin{equation}
      p(\mathbf{x}) = p(\mathbf{x}_b | \mathbf{x}_a)p(\mathbf{x}_a) = \frac{(2 \pi)^{-\left(n_{a}+n_{b}\right) / 2}}{\sqrt{\det \boldsymbol{\Sigma}_{b | a} \det \boldsymbol{\Sigma}_{a}}} \exp \left\{ -\frac{1}{2} E \right\}
    \end{equation}

    where 

    \begin{equation}
      E=\left(\mathbf{x}_{b}-\mathbf{M} \mathbf{x}_{a}-\mathbf{d}\right)^\top \boldsymbol{\Sigma}_{b | a}^{-1}\left(\mathbf{x}_{b}-\mathbf{M} \mathbf{x}_{a}-\mathbf{d}\right)+\left(\mathbf{x}_{a}-\boldsymbol{\mu}_{a}\right)^\top \boldsymbol{\Sigma}_{a}^{-1}\left(\mathbf{x}_{a}-\boldsymbol{\mu}_{a}\right).
    \end{equation}

    Rewriting $\mathbf{x}_{b}-\mathbf{M} \mathbf{x}_{a}-\mathbf{d}$ as $f$ and $\mathbf{x}_{a}-\boldsymbol{\mu}_{a}$ as $e$, we have

    \begin{align}
      \label{eq:app-affine-transf-cov-matrix}
      E&=\left(f-\mathbf{M}e\right)^\top \boldsymbol{\Sigma}_{b | a}^{-1}\left(f-\mathbf{M}e\right)+\left(e\right)^\top \boldsymbol{\Sigma}_{a}^{-1}\left(e\right) \nonumber \\
      &= e^\top \left(\mathbf{M}^\top  \boldsymbol{\Sigma}_{b | a}^{-1} \mathbf{M}+\boldsymbol{\Sigma}_{a}^{-1}\right) e-e^\top  \mathbf{M}^\top  \boldsymbol{\Sigma}_{b | a}^{-1} f-f \boldsymbol{\Sigma}_{b | a}^{-1} \mathbf{M} e+f^\top  \boldsymbol{\Sigma}_{b | a}^{-1} f \nonumber \\
      &=\left(\begin{array}{c}{e} \\ {f}\end{array}\right)^\top \left(\begin{array}{cc}{\mathbf{M}^\top  \boldsymbol{\Sigma}_{b | a}^{-1} \mathbf{M}+\boldsymbol{\Sigma}_{a}^{-1}} & {-\mathbf{M}^\top  \boldsymbol{\Sigma}_{b | a}^{-1}} \\ {-\boldsymbol{\Sigma}_{b | a}^{-1} \mathbf{M}} & {\boldsymbol{\Sigma}_{b | a}^{-1}}\end{array}\right)\left(\begin{array}{l}{e} \\ {f}\end{array}\right) \nonumber  \\
      &= \left(\begin{array}{c}{\mathbf{x}_{a}-\boldsymbol{\mu}_{a}} \\ {\mathbf{x}_{b}-\mathbf{M} \boldsymbol{\mu}_{a}-\mathbf{d}}\end{array}\right)^\top \mathbf{R}^{-1}\left(\begin{array}{c}{\mathbf{x}_{a}-\boldsymbol{\mu}_{a}} \\ {\mathbf{x}_{b}-\mathbf{M} \boldsymbol{\mu}_{a}-\mathbf{d}}\end{array}\right)
    \end{align}

    By (\ref{eq:app-det-schur-complement}) we have that

    \begin{align}
      \det \mathbf{R}^{-1}&=\det\left(\boldsymbol{\Sigma}_{b | a}^{-1}\right) \det\left(\mathbf{M}^\top \boldsymbol{\Sigma}_{b | a}^{-1} \mathbf{M}+\boldsymbol{\Sigma}_{a}^{-1}-\mathbf{M}^\top \boldsymbol{\Sigma}_{b | a}^{-1} \boldsymbol{\Sigma}_{b | a} \boldsymbol{\Sigma}_{b | a}^{-1} \mathbf{M}\right) \nonumber  \\
      &=\det\left(\boldsymbol{\Sigma}_{b | a}^{-1}\right) \det\left(\boldsymbol{\Sigma}_{a}^{-1}\right) \nonumber  \\
      &=\frac{1}{\det\left(\boldsymbol{\Sigma}_{b | a}\right) \det\left(\boldsymbol{\Sigma}_{a}\right)}
    \end{align}
    
    Finally by inspection of (\ref{eq:app-affine-transf-cov-matrix}), we take the mean of $\mathbf{x}_b$ as $\mathbf{M} \boldsymbol{\mu}_{a}+\mathbf{d}$ which concludes the proof.
  \end{proof}

  With the results of Theorems 1, 2 and 3 we get the following corollary

  \begin{corollary}

    Being $\mathbf{x}_b$ conditioned on $\mathbf{x}_a$ and Gaussian distributed as

    \begin{equation}
      p\left(\mathbf{x}_{a}\right)=\mathcal{N}\left(\mathbf{x}_{a} | \boldsymbol{\mu}_{a}, \boldsymbol{\Sigma}_{a}\right), \quad p\left(\mathbf{x}_{b} | \mathbf{x}_{a}\right)=\mathcal{N}\left(\mathbf{x}_{b} | \mathbf{M} \mathbf{x}_{a}+\mathbf{d}, \boldsymbol{\Sigma}_{b | a}\right)
    \end{equation}

    with $\boldsymbol{\mu}_{b} =M \boldsymbol{\mu}_{a}+d$, $\boldsymbol{\Sigma}_{b} =\boldsymbol{\Sigma}_{b | a}+\mathbf{M} \boldsymbol{\Sigma}_{a} M^{T}$, $\mathbf{M}$ a constant matrix and $\mathbf{d}$ a constant vector, both with the appropriate dimensions. Then conditional distribution $p(\mathbf{x}_a|\mathbf{x}_b)$ is given by

    \begin{subequations}
    
    \begin{align}
      p\left(\mathbf{x}_{a} | \mathbf{x}_{b}\right)&=\mathcal{N}\left(\mathbf{x}_{a} | \boldsymbol{\mu}_{a | b}, \boldsymbol{\Sigma}_{a | b}\right)
    \end{align}
    with
    \begin{align}
        \boldsymbol{\mu}_{a | b}&=\boldsymbol{\Sigma}_{a | b}\left(\mathbf{M}^\top \boldsymbol{\Sigma}_{b | a}^{-1}\left(\mathbf{x}_{b}-\mathbf{d}\right)+\boldsymbol{\Sigma}_{a}^{-1} \boldsymbol{\mu}_{a}\right) \\ \boldsymbol{\Sigma}_{a | b}&=\left(\boldsymbol{\Sigma}_{a}^{-1}+\mathbf{M}^\top \boldsymbol{\Sigma}_{b | a}^{-1} \mathbf{M}\right)^{-1}.
    \end{align}
  \end{subequations}
  \end{corollary}

\end{appendices}

\newpage
\bibliography{Intro_to_GP_bibliography}

\end{document}